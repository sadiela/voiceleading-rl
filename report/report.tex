\documentclass[12pt, letterpaper]{article}
\usepackage[english]{babel}
\usepackage{parskip}
\usepackage{amsmath}
\usepackage{amsfonts}

\title{
    EC700 Project Report \\
    Reinforcement Learning on Harmonic Progression
}
\author{Sadie Allen, Chonghua Xue, Liam Ramsay}
\date{May 1st, 2023}

\begin{document}
\maketitle
\newpage

\section{Introduction}

The intersection of music theory and artificial intelligence has long been a fascinating area of study, with researchers consistently uncovering novel ways to analyze, compose, and understand music. In this report, we explore a unique approach to this interdisciplinary challenge, focusing on the application of reinforcement learning to harmonic progression in music theory. By examining the complex relationships between chords and progressions, we aim to create a computational model that can learn, predict, and generate harmonic sequences using the principles of reinforcement learning.

Harmonic progression lies at the heart of Western music, providing the foundation for the listener's sense of tension and resolution, as well as contributing to the overall emotional impact of a piece. Traditionally, music theorists have devised rules and guidelines for creating harmonically pleasing progressions, which are based on centuries of musical practice and observation. However, these rules are often complex and difficult for both humans and machines to navigate. In this project, we seek to teach an artificial intelligence agent the principles of harmonic progression through the use of reinforcement learning, allowing it to autonomously learn and adapt its knowledge of these rules as it interacts with its environment.

By leveraging reinforcement learning, our project aims to develop a model that can intelligently navigate the intricate world of harmonic progressions, learning from experience and refining its understanding of music theory over time.

This report will detail the methodology, implementation, and results of our reinforcement learning approach to harmonic progression, as well as discuss the broader implications of our findings for the future of music composition and analysis. We will begin by providing an overview of the relevant background in music theory and reinforcement learning, followed by an in-depth explanation of our proposed model and its implementation. Finally, we will present the results of our experiments and discuss their implications for the field of music theory and artificial intelligence.

\section{Harmonic Progression}

\section{Methods}

\section{Results}

\section{Discussions}

\end{document}